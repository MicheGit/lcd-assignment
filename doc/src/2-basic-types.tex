\section{Types}

\subsection{Basic types}

\begin{frame}{Basic Types - Syntax}
    % TODO split in due colonne, riportare in una quella originale di vasconcelos e nell'altra quella implementata
    Implementation of qualifier, pretypes and types are fully compliant to Chapter 3's original syntax:
    \begin{minipage}{\textwidth}
        \begin{minipage}{0.45\textwidth}
            \begin{flalign*}
                q ::= & lin \\
                & un \\
                p ::= & ?T.T \\
                & !T.T \\
            \end{flalign*}
        \end{minipage}
        \begin{minipage}{0.45\textwidth}
            \begin{flalign*}
                T :== & bool \\
                & end \\
                & q p \\
                \Gamma ::= & \emptyset \\
                & \Gamma , x : T 
            \end{flalign*}
        \end{minipage}
    \end{minipage}
    Contexts are implemented as hash maps of types
\end{frame}

\begin{frame}[fragile]{Basic Types - Duality}
    Duality is partially defined as follows:
    \begin{minted}{hs}
    dualType :: SpiType -> SpiType
    dualType End = End
    dualType Boolean = error "..."
    dualType (Qualified q (Receiving t1 t2)) = 
        Qualified q (Sending t1 (dualType t2))
    dualType (Qualified q (Sending t1 t2)) = 
        Qualified q (Receiving t1 (dualType t2))
    dualType (Recursive a p) = Recursive a (dualType p)
    dualType (TypeVar x) = TypeVar 
    \end{minted}
\end{frame}

\begin{frame}[fragile]{Basic Types - Context}
    Contexts support the following operations:
    \begin{itemize}
        \item (nondeterministic) \textbf{context split} $\Gamma = \Gamma_1 \circ \Gamma_2$: the function \texttt{ndsplit :: Context -> [(Context, Context)]} creates all possible combinations of dividing \texttt{lin}ear variables, mantaining \texttt{un}restricted variables;
        \item \textbf{update} $\Gamma + (x : T)$: the function \texttt{update k t} inserts $k : t$ in the context only if the variable $k$ was not present, or it was yet defined \texttt{un}restricted with type $t$;
        \item \textbf{override} $\Gamma, (x : T)$: this represents newly bounded variables, possibly shadowing preexisting definitions.
    \end{itemize}
\end{frame}

\begin{frame}[fragile]{Basic Types - Typing Rules (1)}

    Sequent calculus rules are modeled as instnces of the context transition \textbf{monad}:

    \begin{minted}{hs}
newtype CT a = CT 
    (Context -> TypeErrorBundle TypeError (a, Context))

instance Monad CT where
    return :: a -> CT a
    (>>=) :: CT a -> (a -> CT b) -> CT b
\end{minted}
    Rules can be composed and propagate context side effects

    A check rule can either hold and return \texttt{()} or fail and return an error:
    \begin{minted}{hs}
class TypeCheck a where
    check :: a -> CT ()
    \end{minted}

\end{frame}

\begin{frame}[fragile]{Basic Types - Typing Rules (2)}

    \begin{exampleblock}{Unrestricted requirement}
        $\texttt{un}(\Gamma)$ holds when all entries in the context are unrestricted:
\vspace{0.5cm}
        \begin{minted}{hs}
unGamma :: CT ()
unGamma = CT (\context -> if all unrestricted context
    then Right ()
    else Left "Error message...")  
        \end{minted}
    \end{exampleblock}
\end{frame}

\begin{frame}[fragile]{Basic Types - Typing Rules (3)}

    \begin{exampleblock}{Context update}
        $\Gamma + (x : T)$ throws an error if the conditions arent met:
        \small\begin{minted}[escapeinside=//,mathescape=true]{hs}
update :: String -> SpiType -> CT ()
update k t = do
    may <- liftPure (M.lookup k)
    case may of
        Just found -> unless (predicate Un t && found /$\approx$/ t) 
            (throwError "Error message..")
        Nothing    -> sideEffect (M.insert k t)
        \end{minted}
    \end{exampleblock}
    
\end{frame}


\begin{frame}[fragile]{Basic Types - Typing Rules (4)}

    \begin{exampleblock}{Context update}
        Context transitions can even return useful values:
        \small\begin{minted}{hs}
extract :: String -> CT SpiType
extract k = do
    t <- get k
    unless (predicate Un t) (delete k)
    return t
        \end{minted}
    \end{exampleblock}

    This is used to optimize some context splits in the [\textsc{T-Rec}] and [\textsc{T-Send}] rules

\end{frame}

% \begin{frame}{Language - Precomputation}
%     After parsing, the program is precomputed lifting all bindings over all parallel compositions, as expressed in the following \textbf{structural congruence}: 
%     \[ (\nu x y) P | Q \equiv (\nu x y) (P | Q) \]
% \end{frame}

% \begin{frame}[fragile]{Language - Operational Semantics}
%     The code interpreter prints debug information about the program executed, along with a timestamp and a description of what the process did at that time.

%     \begin{block}{Output format}
%         \begin{lstlisting}
% [TIMESTAMP | ThreadId THREAD_ID]: MESSAGE
%         \end{lstlisting}
%     \end{block}

%     Program behaviour is defined according to the \textbf{operational semantics} presented in chapter 2.
% \end{frame}

% \subsection{Runtime}

% \begin{frame}{Runtime - Concurrent Haskell}

%     The language runtime is implemented in concurrent Haskell, using:
%     \begin{itemize}
%         \item the \texttt{IO} monad, along with a local state that maps variables to channels and literals, to model threads;
%         \item and the \texttt{MVar} type to model channels.
%     \end{itemize}

%     Each thread is created with the \[ \texttt{forkIO :: IO () -> IO ThreadId} \] function, that simply creates a new thread, returning its id. % NO default AWAIT
    
% \end{frame}

% \begin{frame}{Runtime - Channels (1/2)}
%     Plain \texttt{MVar}s can await for value insertion with the function \[ \texttt{takeMVar :: MVar a -> IO a} \] but the dual operation \[ \texttt{putMVar :: MVar a -> a -> IO ()} \] does not await the variable to be ready to accept a new value.
% \end{frame}


% \begin{frame}{Runtime - Channels (2/2)}
%     Channels are represented as tuples of \texttt{MVar}, meaning \texttt{(value, idle)}:
%     \begin{itemize}
%         \item \texttt{value :: MVar }$v$, containing the passed value, which needs to be evaluated;
%         \item \texttt{idle :: MVar ()}, which has a \texttt{()} value in it when the channel is ready to receive.
%     \end{itemize}
%     Then, the \texttt{Channel} datatype further distinguishes whether it is a read end or a write end. Read ends and write ends share the same \texttt{MVar}s.
% \end{frame}

% \begin{frame}{Runtime - Program Behaviour (1/3)}
%     \begin{block}{Inaction}
%         The process $0$ just prints \texttt{STOP} and ends the thread.
%     \end{block}

%     \begin{block}{Branching}
%         The process $\texttt{if }v\texttt{ then }P_1\texttt{ else }P_2$ prints two debug messages, both starting with \texttt{BRANCHING}:
%         \begin{itemize}
%             \item the guard before evaluation;
%             \item and then, after evaluation in the local state.
%         \end{itemize}
%         After that, the run continues as the appropriate process.
%     \end{block}
% \end{frame}

% \begin{frame}{Runtime - Program Behaviour (2/3)}
%     \begin{block}{Binding}
%         The process $(\nu x y) . P$ just prints \texttt{BINDING} followed by the two bounded variables. Then, it creates two \texttt{MVar}s, one for the value and one for the lock and associates in the local state both variables with the respective ends.

%         The run proceeds in the same thread as prescribed by $P$.
%     \end{block}

%     \begin{block}{Fork}
%         The process $P_1 | P_2$ prints \texttt{FORK} followed by the two new processes ids.

%         To prevent a concurrent program to end before all forked threads terminate, this code constructs two \texttt{MVar}s that are notified when the two new threads finish. This process doesn't do anything more than awaiting for both threads to finish.
%     \end{block}
% \end{frame}

% \begin{frame}{Runtime - Program Behaviour (3/3)}
%     \begin{block}{Sending}
%         The process $\overline{x} v . P$ prints \texttt{SENDING} followed by the value: 
%         \begin{itemize}
%             \item before evaluation;
%             \item and after evaluation.
%         \end{itemize}
        
%         Then the process sends the value over the channel $x$ and proceeds as prescribed by P.
%     \end{block}

%     \begin{block}{Receiving}
%         The process $x(v) . P$ prints \texttt{RECEIVING} followed by: 
%         \begin{itemize}
%             \item the newly bound variable name;
%             \item and the value received.
%         \end{itemize}
        
%         Then the process proceeds as prescribed by P.
%     \end{block}
% \end{frame}