\section{Algorithmic Type Checking}

\begin{frame}{Algorithmic Type Checking (1/3)}
    The problem with the parallel composition rule is that we can't know a priori which parallel branch uses a variable.

    \vspace{0.5cm}

    However, we can easily avoid this problem by keeping track of \textbf{variables} used in \textbf{subject position}, such as channel send or receive.

    \vspace{0.5cm}

    For processes, output context $\Gamma_2$ contains all free variables ``consumed'' by process $P$, $L$ contains all linear free variable used in process $P$
    \begin{flalign*}
        & & \Gamma_1 \vdash P : \Gamma_2 ; L & & \Gamma_1 \vdash x : T ; \Gamma_2 & &
    \end{flalign*}

    For variables, the output is a context without the variable, if it was linear

\end{frame}

\begin{frame}[fragile]{Algorithmic Type Checking (2/3)}

    Then \textbf{context difference} $\Gamma \div L$ removes all variables in $L$ from $\Gamma$ and yields an error if there was a linear variable in $L$.

    \vspace{0.5cm}

    Then, parallel composition rule is easily defined:
    \begin{flalign*}
        \begin{prooftree}
            \hypo{\Gamma_1 \vdash P : \Gamma_2 ; L_1}
            \hypo{\Gamma_2 \div L_1 \vdash Q : \Gamma_3 ; L_2}
            \infer2[[\textsc{A-Par}]]{\Gamma_1 \vdash P | Q : \Gamma_3 ; L_2}
        \end{prooftree}
    \end{flalign*}
    The implementation is straight forward:
    \begin{minted}{hs}
        check (Par p1 p2) = do
            l1 <- check p1
            contextDiff l1
            check p2
    \end{minted}
\end{frame}

\begin{frame}[fragile]{Algorithmic Type Checking (3/3)}
    
    Other rules have similar implementation. The optimizations for direct context split check of [\textsc{T-In}] and [\textsc{T-Out}] are related.

    \vspace{0.5cm}

    Rule [\textsc{A-If}] requires that the outputs for both branches are the same:
    \begin{flalign*}
        \begin{prooftree}
            \hypo{\Gamma_1 \vdash v : q \text{ bool} ; \Gamma_2}
            \hypo{\Gamma_2 \vdash P : \Gamma_3 ; L}
            \hypo{\Gamma_2 \vdash Q : \Gamma_3 ; L}
            \infer3[[\textsc{A-If}]]{\Gamma_1 \vdash \text{if }v\text{ then }P\text{ else }Q: \Gamma_3 ; L}
        \end{prooftree}
    \end{flalign*} 

\end{frame}